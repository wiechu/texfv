%
% Z podziękowaniem dla p.Jarka – największego specjalisty od TeXa
% na wszystkich kontynentach z wykluczeniem tego,
% na którym mieszka Donald Knuth

%
% Wersja z groszami w cenie netto (trzeba podawać cenę w groszach)
%

\nopagenumbers
\vsize11in
\voffset-1cm

% fonty
%\font\ss=phvro8tn at 6pt
%\font\sm=phvr8tn at 8pt
%\font\sn=phvr8tn at 10pt
%\font\sf=phvr8t at 10pt
%\font\Sf=phvr8t scaled \magstep1
%\font\SF=phvr8t scaled \magstep2

\def\Font{Nimbus Sans Narrow}

\font\ss="\Font/I" at 6pt
\font\sm="\Font:extend=0.82" at 8pt
\font\sn="\Font:extend=0.82" at 10pt
\font\sf="\Font:extend=0.82" at 10pt
\font\Sf="\Font:extend=0.82" scaled \magstep1
\font\SF="\Font:extend=0.82" scaled \magstep2

%\font\ss="Roboto Condensed/I" at 5.5pt
%\font\sm="Roboto Condensed:style=Regular" at 8pt
%\font\sn="Roboto Condensed:style=Regular" at 9.2pt
%\font\sf="Roboto Condensed:style=Regular" at 10pt
%\font\Sf="Roboto Condensed:style=Regular" scaled \magstep1
%\font\SF="Roboto Condensed:style=Regular" scaled \magstep2

%\font\ss="[lmsans8-oblique]" at 6pt
%\font\sm="[lmsans8-regular]" at 8pt
%\font\sn="[lmsans8-regular]" at 10pt
%\font\sf="[lmsans8-regular]" at 10pt
%\font\Sf="Latin Modern Roman/S=10" scaled \magstep1
%\font\SF="Latin Modern Roman/S=10" scaled \magstep2

% ########### Nagłówek (Sprzedawca/Odbiorca/Daty)
% 
% Elementy nagówka
%
\newdimen\H \H=2.5cm
\newdimen\G \G\H \advance\G by 6pt
\def\frSprzedawca {\pudloD 5.2cm {\H} Sprzedawca: {\sn\MojaFirma}}
\def\frNabywca    {\pudloD 7.3cm {\H} Nabywca:    {\sn\dlakogo}}
\def\frDaty	  {\vbox to \G {
		    \pudloA 3cm .6cm {Miejscowość:} {\sn \miejscowosc}\vfil
		    \pudloA 3cm .6cm {Data wystawienia faktury:} {\sn \the\day\ \mies\ \the\year}\vfil
		    \pudloA 3cm .6cm {Data sprzedaży:} {\sn \the\day\ \mies\ \the\year}}}
%
% Pierwsza ramka
%
\def\fvatHead#1 { % wołane z parametrem 'Oryginał' albo 'Kopia'
\line{\frSprzedawca\hfil\frNabywca\hfil\frDaty\break}
\vskip 3pt
\hrule\hbox{\vrule\vbox{\vskip7pt
\line{\SF\hfill Faktura VAT Nr \nrfaktury\hfill}
\vskip2pt
\line{\sf\hfill #1 \hfill}
\vskip3pt}\vrule}\hrule
} %end of fvatHead


%
% %%%%%%%%%%%%%%% Pozycje faktury
%
% Szerokość dwóch ostatnich kolumn tabeli
\newdimen\R \R=60pt
\newdimen\dwaR \dwaR=120pt
\newdimen\trzyR \trzyR=180pt

%
% Elementy tabelek sprzedaży towarów
%
\def\strut{\vrule height 12pt depth 6pt width 0pt}
\def\vf#1{\kern3pt\sf\strut #1\hfill}
%\def\vf#1{\kern3pt\sf\strut\vbox{\hsize5cm\parindent0pt \vss#1\vss}\hfill}
\def\vl#1#2{\hbox to #1{\strut\kern3pt#2\hfill\kern3pt}}
\def\vc#1#2{\hbox to #1{\strut\kern3pt\hfill#2\hfill\kern3pt}}
\def\vr#1#2{\hbox to #1{\strut\kern3pt\hfill#2\kern3pt}}

\def\LongTowarName#1{\vbox{\hsize 9cm \vskip 2mm \noindent #1}}

%
% Nagłówek tabeli sprzedaży towarów
%
\def\tableHead{\sf
\vskip3pt
\nettoA=0
\nettoB=0
\nettoZ=0
\Lp=0
\hrule
\line{\vrule
\vc{7mm}{\sm Lp.}\vrule
\vf{\sm Nazwa towaru lub usługi}\vrule
\vc{24pt}{\sm Ilość}\vrule
\vc{24pt}{\sm j.m.}\vrule
\vc{13pt}{\ss \%}\vrule
\vc{\R}{\sm Cena}\vrule
\vc{\R}{\sm Wartość netto}\vrule
}\hrule}

%
% Kolejne linie z towarem
%
\def\liniaTowaru#1#2#3#4#5{%
\advance\Lp 1
\line{\vrule
\vr{7mm}{\the\Lp.}\vrule	% Lp.
\vf{\LongTowarName{#1}}\vrule			% Nazwa
\vr{24pt}{#2}\vrule		% Ilość
\vc{24pt}{\sm #4}\vrule	% j.m
\vc{13pt}{\ss #5}\vrule		% stawka
\vr{\R}{\PLN{\cena}}\vrule	% Cena
\vr{\R}{\PLN{\aIxC}}\vrule	% Wartość netto
}\hrule}

%
% Towary 23%
%
\def\artA#1 | #2 | #3 | #4 {
\cena=#3 \advance\cena by 0
\aIxC\cena \multiply\aIxC by #2
\advance\nettoA\aIxC
\liniaTowaru{#1}{#2}{#3}{#4}{23}
}
%
% Towary 8%
%
\def\artB#1 | #2 | #3 | #4 {
\cena=#3 \advance\cena by 0
\aIxC\cena \multiply\aIxC by #2
\advance\nettoB\aIxC
\liniaTowaru{#1}{#2}{#3}{#4}{8}
}

%
% Towary zw. (zwolnione z VAT)
%
\def\artZ#1 | #2 | #3 | #4 {
\cena=#3 %\multiply\cena by 100
\advance\cena by 0
\aIxC\cena \multiply\aIxC by #2
\advance\nettoZ\aIxC
\liniaTowaru{#1}{#2}{#3}{#4}{zw.}
}


%
% Cały dół ze słownie i takie tam dla jednej stawki
%
\def\sumaJednaStawka#1{
\line{\vbox{\hsize 121mm%\hrule - jak się chce zobaczyć szerokość tej skrzynki
\noindent\Sf Do zapłaty: {\SF\PLN{\sumaBrutto}}
\vskip2mm
\noindent{\sf Słownie: \slownieZlGr}
\vskip2mm
\vfil
\ifbank{\vbox{\noindent\Sf Płatne przelewem w terminie \dni\ dni na konto:\par\noindent\konto}}%
\else{\noindent\Sf Zapłacono gotówką}\fi\vfil}
\hfil\vrule\vbox{\hrule%
\hbox{\vc{\dwaR}{RAZEM}}\hrule
\hbox{\vl{\R}{Netto}\vrule\vr{\R}{\PLN{\sumaNetto}}}\hrule
\hbox{\vl{\R}{VAT #1\%}\vrule\vr{\R}{\PLN{\sumaVat}}}\hrule
\hbox{\vl{\R}{Brutto}\vrule\vr{\R}{\PLN{\sumaBrutto}}}\hrule}\vrule}\vrule
}


%
% Cały dół ze słownie i takie tam dla stawek 8 i 23
%
\def\sumaRozbiteStawki{
\line{\vbox{\hsize 76mm%
\noindent\Sf Do zapłaty: {\SF\PLN{\sumaBrutto}}
\vskip2mm
%\hsize 76mm
\noindent{\sf Słownie: \slownieZlGr}
\vskip5mm
\vfil
\noindent%
\ifbank{\vbox{\noindent\sf Płatne przelewem w terminie \dni\ dni na konto:}}%
\else{\noindent Zapłacono gotówką}\fi
\vfil}
\hfil\vrule\vbox{\hrule%
\hbox{\vr{\R}{\sn Stawka VAT}\vrule\vr{\R}{Netto}\vrule\vr{\R}{VAT}\vrule\vr{\R}{Brutto}}\hrule
\ifnum0<\nettoA \hbox{\vr{\R}{23\%}\vrule\vr{\R}{\PLN{\nettoA}}\vrule\vr{\R}{\PLN{\vatA}}\vrule\vr{\R}{\PLN{\bruttoA}}}\hrule\fi
\ifnum0<\nettoB \hbox{\vr{\R}{ 8\%}\vrule\vr{\R}{\PLN{\nettoB}}\vrule\vr{\R}{\PLN{\vatB}}\vrule\vr{\R}{\PLN{\bruttoB}}}\hrule\fi
\ifnum0<\nettoZ \hbox{\vr{\R}{zw. }\vrule\vr{\R}{\PLN{\nettoZ}}\vrule\vr{\R}{\PLN{    0}}\vrule\vr{\R}{\PLN{\nettoZ }}}\hrule\fi
\vskip1pt\hrule
\hbox{\vl{\R}{Razem}\vrule\vr{\R}{\PLN{\sumaNetto}}\vrule\vr{\R}{\PLN{\sumaVat}}\vrule\vr{\R}{\PLN{\sumaBrutto}}}\hrule}\vrule}
%
\line{\vbox{\hsize 10in
\vfil
\Sf\vbox{\noindent\ifbank{\vbox{\noindent \konto}}\fi}}}
}


% %%%%%%%%%%%%%%%%%%%%%%%%%%%% Podsumowanie 
\def\fvatFoot{
\vskip3pt
%
% Obliczamy sumy końcowe
%
% bruttoA = 1.23 * nettoA
\bruttoA\nettoA \multiply\bruttoA 123 \advance\bruttoA50 \divide\bruttoA100
% vatA = bruttoA - nettoA
\vatA\bruttoA \advance\vatA-\nettoA
% bruttoB = 1.08 * nettoB
\bruttoB\nettoB \multiply\bruttoB 108 \advance\bruttoB50 \divide\bruttoB100
% vatB = bruttoB - nettoB
\vatB\bruttoB \advance\vatB-\nettoB
% sumaNetto = nettoA + nettoB + nettoZ
\sumaNetto\nettoA \advance\sumaNetto by \nettoB \advance\sumaNetto\nettoZ
% sumaVat = vatA + vatB
\sumaVat\vatA \advance\sumaVat by \vatB
% sumaBrutto = sumaNetto + sumaVat
\sumaBrutto\sumaNetto \advance\sumaBrutto by \sumaVat
%\zl i \gr To jest do napisania "słownie"
\zl=\sumaBrutto \divide\zl by 100 \gr=\sumaBrutto
\zlgr=\zl \multiply\zlgr by 100%
\advance\gr by -\zlgr

\tolerance=10000% Jakby się słownie miało nieprzenosić czy co tam
\ifnum\sumaNetto=\nettoA
    \sumaJednaStawka{23}
\else
    \ifnum\sumaNetto=\nettoB
        \sumaJednaStawka{8}
    \else
        \sumaRozbiteStawki
    \fi
\fi
\vskip3mm\fvPodpisy%
}

%
% podpisy pod fakturą
%

\def\fvPodpisy{
\line{%
\pudloD 7cm 1.5cm {Osoba upoważniona do odbioru faktury VAT:} {\sm \klient}%
\hfill%
\pudloD 7cm 1.5cm {Osoba upoważniona do wystawienia faktury VAT:} {\sm \wystawil}%
}}


%
% Używane liczniki
%
\newcount\Lp
\newcount\cena
\newcount\aIxC

\newcount\nettoA
\newcount\nettoB
\newcount\nettoZ

\newcount\vatA
\newcount\vatB

\newcount\bruttoA
\newcount\bruttoB

\newcount\sumaNetto
\newcount\sumaVat
\newcount\sumaBrutto

%%%%%%%%%%%%%%%% Takie tam różne moje definicje %%%%%%%%%%%%%%%%%%%%
\def\KilkaLinii#1{\def\\{\par}\parindent0pt#1}

%%%%%%%%%%%%%%%%
\def\MojaFirma{} \def\sprzedawca#1\par{\def\MojaFirma{\KilkaLinii#1}}
\def\dlakogo{}   \def\dla#1\par{\gdef\dlakogo{\def\\{\par}\parindent0pt#1}}
\def\nrfaktury{} \def\faktura#1 {\gdef\nrfaktury{#1}}
\def\klient{} \def\kto#1\par{\gdef\klient{#1}}
\def\wystawil{} \def\mojpodpis#1\par{\gdef\wystawil{#1}}
\def\miejscowosc{} \def\gdzie#1\par{\gdef\miejscowosc{#1}}
\def\dnia#1.#2.#3 {\year#1 \month#2 \day#3 \ifnum\year<100\advance\year2000\fi}
\newif\ifbank \bankfalse
\def\bank#1 {\banktrue\gdef\dni{#1}}
\def\banknot{\bankfalse}

\def\mies{\ifcase\month\or
stycznia\or lutego\or marca\or kwietnia\or maja\or czerwca\or lipca\or
sierpnia\or września\or października\or listopada\or grudnia\fi}

% Różne pudła
\def\boxit#1{\vbox{\hrule\hbox{\vrule\kern3pt\vbox{\kern3pt #1\kern3pt}\kern3pt\vrule}\hrule}}

\def\pudloA#1 #2 #3 #4{\boxit{\vbox to #2 {\hsize #1
\noindent
\leftskip0pt \rightskip0pt plus 1fill
\ss #3\vss
\leftskip0pt plus 1fill \rightskip0pt
\noindent{#4}
\vss}}}

\def\pudloD#1 #2 #3 #4{\boxit{\vbox to #2 {\hsize #1
\noindent
\leftskip0pt \rightskip0pt plus 1fill
\ss #3\vss
\leftskip0pt plus 1fill \rightskip0pt plus 1fill
\noindent{#4}
\vss}}}

% Różne makra używane do pisania pozycji
\newcount\zl
\newcount\gr
\newcount\zlgr

\def\PLN#1{{\zl=#1 \gr=\zl
\divide\zl by 100 \the\zl,%
\zlgr=\zl \multiply\zlgr by 100%
\advance\gr by -\zlgr
\ifnum10>\gr 0\fi\the\gr\ zł}}
                                        
% %%%%%%%%%%%%%%% Makra piszące liczbę słownie
\def\Ni#1{\ifcase#1\or
jeden    \or dwa      \or trzy     \or cztery   \or pięć     \or
sześć    \or siedem   \or osiem    \or dziewięć \or dziesięć \or
jedenaście \or dwanaście    \or trzynaście \or czternaście \or piętnaście \or 
szesnaście \or siedemnaście \or osiemnaście \or dziewiętnaście \fi}
 
\def\Nii#1{\ifcase#1\or
dziesięć \or dwadzieścia \or trzydzieści \or czterdzieści \or pięćdziesiąt \or
sześćdziesiąt \or siedemdziesiąt \or osiemdziesiąt \or dziewięćdziesiąt \fi}
 
\def\Niii#1{\ifcase#1\or
sto \or dwieście \or trzysta \or czterysta \or pięćset 
\or sześćset \or siedemset \or osiemset \or dziewięćset \fi}
 
\def\Ntys#1{\ifcase\count12\or
tysiąc \or
tysiące \or
tysiące \or
tysiące \else
tysięcy \fi}

\def\SS#1{{\count0=#1%
\count3=\count0 
          \divide\count3 by 100
          \count13\count3
          \multiply\count3 by 100
\count2=\count0 
          \advance\count2 by -\count3
          \divide\count2 by 10
          \count12\count2
          \multiply\count2 by 10
\count1=\count0 
          \advance\count1 by -\count2
          \advance\count1 by -\count3
\ifnum1=\count12 \count12=0 \advance\count1 by 10 \fi
\Niii{\count13}%
\Nii{\count12}%
\Ni{\count1}%
}}
 
\def\slownie#1{{\count0=#1%
\count3=\count0
          \divide\count3 by 1000000
          \count13\count3
          \multiply\count3 by 1000000
\count2=\count0
          \advance\count2 by -\count3
          \divide\count2 by 1000
          \count12\count2
          \multiply\count2 by 1000
\count1=\count0
          \advance\count1 by -\count2
          \advance\count1 by -\count3
          \count11\count1
\ifnum\count13>0 \SS{\count13}mln. \fi
\ifnum\count12>0 \SS{\count12}\Ntys \fi
\SS{\count11}%
}}

\def\Nzl#1{{%
\count0=#1
\count1=\count0
\divide\count1 by 100
\multiply\count1 by 100
\advance\count0 by -\count1
\ifcase\count0
złotych\or	% 0
złotych\or	% 1
złote\or	% 2
złote\or	% 3
złote\else	% 4
\ifnum\count0<22
złotych
\else
{\count0=#1
\count1=\count0
\divide\count1 by 10
\multiply\count1 by 10
\advance\count0 by -\count1
\ifcase\count0
złotych\or	% x0
złotych\else	% x1
\Nzl{\count0}
\fi
}
\fi
\fi
}}

\def\Ngr#1{{\count0=#1
\count1=\count0
\divide\count1 by 100
\multiply\count1 by 100
\advance\count0 by -\count1
\ifcase\count0
groszy\or	% 0
grosz\or	% 1
grosze\or	% 2
grosze\or	% 3
grosze\else	% 4
\ifnum\count0<22
groszy
\else
{\count0=#1
\count1=\count0
\divide\count1 by 10
\multiply\count1 by 10
\advance\count0 by -\count1
\ifcase\count0
groszy\or	% x0
groszy\else	% x1
\Ngr{\count0}
\fi
}
\fi
\fi
}}


\def\slownieZlGr{%
\slownie{\the\zl}\Nzl{\the\zl} \ifnum\gr>0 \slownie{\the\gr}\Ngr{\the\gr}\fi
}

%\let\END\end

\newbox\OUT

\def\towar#1\par{\setbox\OUT\vbox{\tableHead#1\fvatFoot}}

\def\Fakturuj{%
\vbox{\fvatHead Oryginał \copy\OUT}\vfill%\eject
%\vbox{\fvatHead Kopia \copy\OUT}\vfil\line{}\vskip 20pt\eject%
\vbox{\fvatHead Kopia \copy\OUT}\vfill\eject%
% Zerowanie po wystawieniu
}

\def\FakturujO{%
\vbox{\fvatHead Oryginał \copy\OUT}\vfill
}

\def\FakturujK{%
\vbox{\fvatHead Kopia \copy\OUT}\vfill
}
