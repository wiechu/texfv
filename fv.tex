%
\input z.inc.core.tex
\input z.inc.user.tex

% Numer faktury
\faktura {123\thinspace{}/\thinspace{}456\thinspace{}/\thinspace{}2020}

% Data faktury (bieżąca data jeśli zakomentowane)
\dnia 2020.06.24

% Dane firmy
\dla Firma\\Spółka z o. o.\\Adres firmy\\NIP:8740343231

% Podpis osoby odbierającej fakturę (można zakomentować)
\kto Nazwisko Prezesa

% sposób zapłaty
%    * gotówka (domyślne): \banknot
%    * przelew: \bank ile_dni_na zapłatę
\bank 10

% Pozycje faktury:
% \art[ABZ] nazwa | ile | cena w groszach | jednostka miary

\towar
\artA Za exploita, który nie działa | 12 | 10000 | szt.
\artB Manualne zaszyfrowanie dysku  | 2  | 25000 | szt.
%\artZ Wywóz komputera na elektrozłom | 13 | 1520 | km


\Fakturuj       % oryginał i kopia na jednej stronie
%\FakturujO     % tylko oryginał
%\FakturujK     % tylko kopia

\bye

% No i teraz `xetex fv.tex`

